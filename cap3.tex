\documentclass[12pt]{article}
\usepackage[portuguese]{babel}
\usepackage{amsmath}
\usepackage{graphicx}
\newcommand{\numpy}{{\tt numpy}}   

\topmargin -.5in
\textheight 9in
\oddsidemargin -.25in
\evensidemargin -.25in
\textwidth 7in

\begin{document}
\textbf{• Problema 3.1} Encontre a entropia $S(E, V, N)$ de um gás ideal de $N$ partículas monoatômicas clássicas, com uma energia total fixa $E$, contido em uma caixa d-dimensional de volume $V$. Deduza a equação de estado deste gás, assumindo que $N$ é muito grande.

\textbf{• Problema 3.2} Considere um gás ideal de $N$ partículas monoatômicas quântico-mecânicas, com uma energia total fixa E, contido em uma caixa hipercúbica d-dimensional, de lado $L$. Para o caso em que E é muito maior do que a energia do estado fundamental, obtenha uma expressão aproximada para a entropia$ S(E, V, N)$ e compare sua expressão com a entropia de um gás clássico. Nesta aproximação, e quando $N$ é grande, qual é a equação de estado do gás quântico? Qual é a probabilidade de encontrar uma partícula com momento $\vec{p}$ neste gás?
\textbf{• Problema 3.4} Para uma coleção de $N$ osciladores harmônicos clássicos tridimensionais de frequência $\omega$ e energia total fixa $E$, calcule a entropia $S$ e a temperatura $T$. Discuta se os osciladores devem ser tratados como distinguíveis ou indistinguíveis.

\textbf{• Problema 3.5} Para uma coleção de $N$ osciladores harmônicos quânticos tridimensionais de frequência $\omega$ e energia total $E$, calcule a entropia $S$ e a temperatura $T$.

\textbf{• Problema 3.6} Usando o ensemble microcanônico, calcule a energia livre de Helmholtz $F(T, N)$ como uma função da temperatura para um sistema de $N$ partículas idênticas, mas distinguíveis, cada uma das quais possui dois níveis de energia. Explore os limites $T \rightarrow 0$ e $T \rightarrow \infty$ da energia, da entropia e dos números de ocupação.

Mostre que a entropia máxima (mínima) corresponde à informação mínima (máxima) sobre o sistema. Quantos bits de informação são perdidos se o sistema evolui de um estado inicial de temperatura zero para um estado final de temperatura infinita? Qual é a temperatura mais alta na qual este sistema pode existir?
\textbf{• Problema 3.9} Considere um sistema de $N$ osciladores harmônicos quântico-mecânicos que não interagem em três dimensões. Calcule a função de partição canônica do sistema $Z(T, N)$. Verifique se a mesma resposta é obtida considerando o sistema como consistindo de:

(a) $3N$ osciladores unidimensionais ou

(b) $N$ osciladores tridimensionais.
\textbf{• Problema 3.18} A uma dada temperatura, a diferença entre os calores específicos de um gás ideal diatômico e um gás monoatômico é devida em parte à energia rotacional das moléculas diatômicas. Um rotor quântico rígido possui níveis de energia $E_\text{rot}(l)$ com degenerescência $g(l)$ dada por 
\[
E_{\text{rot}}(l) = l(l+1) \frac{\hbar^2}{2I} \qquad g(l) = 2l + 1 \qquad l = 0, 1, 2, ...
\]
 onde $l$ é o momento de inércia.
(a) Encontre a função de partição canônica de um gás de N moléculas diatômicas que não interagem.

(b) Avalie o calor específico deste gás em altas temperaturas e nas temperaturas mais baixas em que o movimento rotacional das moléculas dá uma contribuição significativa.

 
 

 
\end{document}